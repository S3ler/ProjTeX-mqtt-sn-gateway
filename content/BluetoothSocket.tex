\section{Bluetooth Low Energy Socket}\label{sec:BluetoothLowEnergySocket}
We wanted to use a low power transmission protocol and we decided to use Bluetooth 4.0 Low Energy (BLE).
One must understand that explaining every detail of the BLESocket implementation would be enough material for a whole new project report.
This is the reason that we focus on the obstacles and provide a overview so that a basic knowledge of the system and the strategies we used are understood.

To understand Bluetooth one must know that BLE knows two types of devices: central and peripheral devices.
Peripheral devices act like server and central devices like clients from a connection oriented perspective\footnote{\url{https://devzone.nordicsemi.com/question/232/what-is-a-client-and-server-in-ble/}, accessed 2017-10-08}.
For example central devices connect to peripheral devices and the peripheral device advertises his presence.
But from a data perspective saying a server contains the data a client requests, then central devices are the servers and peripherals are the clients.
Peripheral device are constrained devices like smart wearables and central devices are unconstrained devices like a Raspberry Pi 3.
So our MQTT-SN devices are peripheral and MQTT-SN gateway are a central devices.

First we give an overview over the problems when trying to program BLE central devices using the Arduino Environment in \autoref{sec:arduinoandble}.
Then we propose our development environment in \autoref{sec:bledevenv}.
Afterwards we show how the basic components of the BLESocket in \autoref{sec:blesocket}.

\subsection{Arduino Environment and BLE}\label{sec:arduinoandble}
Before we started developing the Linux based Bluetooth 4.0 mqtt-sn gateway we looked for a Arduino Board supporting Bluetooth 4.0 as a Central Device.
We found the nRF51822\footnote{\url{www.waveshare.com/wiki/BLE400}, accessed: 2017-10-08} using the Arduino Core for Nordic Semiconductor nRF5 based boards\footnote{\url{https://github.com/sandeepmistry/arduino-nRF5}, accessed: 2017-10-08} and the ESP32\footnote{\url{espressif.com/en/products/hardware/esp32/overview0}, accessed: 2017-10-08} using the Arduino core for ESP32 WiFi chip library\footnote{\url{https://github.com/espressif/arduino-esp32}, accessed: 2017-10-08}.
But both projects do not offer a unified or even any non-dirty solution to program the chips as a Bluetooth 4.0 Central Device.
There was no Bluetooth 4.0 Central library for the Arduino Environment when the project started in 2017-04-01 until we nearly finished in 2017-09-01, too.
We made several attempt and tried out the nRF51822 with some software but creating a own Bluetooth 4.0 Central library for the Arduino Environment is out of the scope of this project.
But for the future the ESP32 seems to be the most promising project as it has a lively discussion around this topic\footnote{\url{https://github.com/espressif/arduino-esp32/issues/423\#issuecomment-331028980}, accessed: 2017-10-08}.
An alternative can be the new released Arduino Primo\footnote{\url{https://store.arduino.cc/arduino-primo}, accessed: 2017-10-08}.
But the release date was somehow never really published so we could not expect that the board will become available in time.
At the end of the project the board was available.
As a result of this whole mess we dropped our attempted to port a Bluetooth 4.0 MQTT-SN gateway to the Arduino Environment and started to program a Linux based version.

\subsection{Development Environment}\label{sec:bledevenv}
As development and test environment we use the hardware and software described in \autoref{sec:developmentenvironment} and \autoref{sec:testenvironment}.
As Bluetooth 4.0 development hardware we use a 7-way hama USB 2.0 Hub with a external power source and four CSL USB Bluetooth Nano-Adapter V4.0 connected to the USB Hub.
A closer look of the devices in appendix \autoref{sec:appendixbledevenv}.
Additionally we use BlueZ version 5.37.
\subsection{BLESocket}\label{sec:blesocket}
For the implementation of the BLESocket class we use the SimpleBluetoothLowEnergySocket which is short discussed in \autoref{sec:simplebluetoothlowenergysocket}. Then we introduce the BLESocket implementation in \autoref{sec:blesocketimplementation}.
\subsubsection{SimpleBluetoothLowEnergySocket}\label{sec:simplebluetoothlowenergysocket}
For implementing the BLESocket we need to write our own SimpleBluetoothLowEnergySocket library\footnote{\url{https://github.com/S3ler/SimpleBluetoothLowEnergySocket}, commit: 0ef008057efd6df3b5d10196bc7397f32b18ee2}.
SimpleBluetoothLowEnergySocket support Central Device use only.
Explaining this library is out of the scope of this project.
But the reasons for creating our own Bluetooth 4.0 Library for Linux are the followings.
Little to no tutorials on how to program Bluetooth LE with C/C++ under Linux are available.
Programming Bluetooth 4.0 directly against the C hci files of the BlueZ library needs root privileges during runtime, this is not acceptable.
Using  Bluetooth 4.0 without root privileges is only possible via the DBUS\footnote{\url{https://www.freedesktop.org/wiki/Software/dbus/}, accessed: 2017-10-08} interface.
The more obstacles appeared.
There is no up-to-date dbus library for C++ available.
Then we needed to use D-Bus Glib\footnote{\url{https://developer.gnome.org/dbus-glib/unstable/}, accessed: 2017-10-08}, because the low-level D-Bus library uses a lot more time to unterstand\footnote{\url{https://dbus.freedesktop.org/doc/api/html/} says: "If you use this low-level API directly, you're signing up for some pain." Accessed: 2017-10-08}.
Then we could not find good tutorials on using Glib, giving us enough information to understand the Glib API Reference and D-Bus at the same time.
Last we obtained the BlueZ source code
\footnote{\url{http://www.bluez.org/release-of-bluez-5-46/}, accessed: 2017-10-08}
 and created a CLion project running the bluetoothctl tool
\footnote{\url{https://github.com/S3ler/bluetoothctl546_experiments}, commit: c25504da0de9742f056094872ab6c31c791cc377}.
With this CLion project, the BlueZ documentation, d-feet\footnote{\url{https://apps.ubuntu.com/cat/applications/d-feet/}, accessed: 2017-10-08} and countless hours of trial and error we were able finally to create the SimpleBluetoothLowEnergySocket library.
\subsection{BLESocket Implementation}\label{sec:blesocketimplementation}
As shown in \autoref{img:blesocketcollaborationdiagram} the BLESocket classes uses four additional classes.
When the BLESocket begin() function is called the BLEConnectionAcceptor is started.
The BLEConnectionAcceptor scans continously for pheripheral of any kind. It runs in a own thread.
When a peripheral is found a new BLEConnection is created by the BLEConnectionAcceptor.
Then the BLEConnection's thread is started.
The BLEConnectionAcceptor now can create the next BLEConnection without blocking.
The BLEConnection started by the BLEConnectionAcceptor tries to connect to the peripheral device.
Successfully connected to the peripheral it check for three special GATT\footnote{\url{https://www.bluetooth.com/specifications/gatt/services}, accessed: 2017-10-08} services\footnote{
	three speical BLE GATT service UUIDs
	\begin{itemize}
		\item UUID: "6E400001-B5A3-F393-E0A9-E50E24DCCA9E"
		\item UUID: "6E400002-B5A3-F393-E0A9-E50E24DCCA9E"
		\item UUID: "6E400003-B5A3-F393-E0A9-E50E24DCCA9E"
	\end{itemize}}.

\image{10cm}{BLESocketCollaborationDiagram.png}{Collaboration diagram of the BLESocket class using the MqttSnMessageHandler}{img:blesocketcollaborationdiagram}
If these services are found the BLEConnection has established a connection to a valid MQTT-SN client via BLE.
MQTT-SN message are received via GATT notifications.

Because all BLEConnections run in a own thread and work completely asynchronously from the rest of the gateway, we somehow need to synchronize the interaction between the BLESocket as part of the MQTT-SN gateway loop() thread and the BLEConnections.
This is implemented as followed.
Each BLEConnection and the BLESocket have access to a shared thread save message queue.
This message queue is filled with instance of the BLEMessage classes when a BLEConnection receives a BLE datagram.
In each loop() call of the BLESocket one BLEMessage is enqueued and consumed by the MqttSnMessageHandler.
This is how we synchronize receiving datagrams from many threaded BLEConnection to one BLESocket.

When the BLESocket shall send data to a MQTT-SN client the device\_address given from the MqttSnMessageHandler is converted to a BLE MAC address.
Now it iterates over all active BLEConnections and sends the matching one the MQTT-SN packet via GATT service.

One problem occur when we always scan for new BLE peripheral devices.
The BLEConnectionAcceptor continously created BLEConnection for not connect able or non MQTT-SN clients (missing GATT services).
To prevent this from happening we manage a set of BLEStatistic by the BLEConnectionAcceptor and so BLE peripheral which are too often unsuccessfully connected are ignored for period of time (for example 10 minutes).
\subsection{Result}
We roughly showed how we can add a new transmission protocol, in this case a Bluetooth 4.0 Low Energy to the existing gateway code.
The implementation of the connection oriented BLESocket is similar to the coding style of a typical TCP server.
Connections are handled in their own thread.
We only needed to synchronize the access to the BLEMessage queue and enqueue messages.
The BLESocket now behaves like a single threaded socket.
The BLEMessage queue bufferes the incoming messages similar to a UDP socket's datagram buffer.