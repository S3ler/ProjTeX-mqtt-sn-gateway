\section{Introduction}\label{sec:introduction}
"By 2020, IoT technology will be in 95\% of electronics for new product designs"\cite{gartnertop10predicitions}.
This means  Internet of Things (IoT) devices will also build with any kind of consumer hardware technology available.
Additionally IoT devices become ubiquitous from small ones like sensors/actuators, smart wearables, smart meters, smart phones, connected vehicles, smart TV, smart furniture\footnote{smart furnitures like smart fridge, toaster and any other you can imagine} and grow to large scale networks over smart homes, smart house infrastructure to smart cities.
We need to bring connectivity into these many different device sizes and potential application areas.
Not only the size of the hardware but also the cost of single device will becomes interesting too.
IoT device vendors are always interested in cheaper devices.
This cost reduction becomes important especially when increasing numbers of devices benefit from.
But smaller and more constrained IoT devices are not only cheaper.
In a lot of cases the smallest, most constrained devices run only with battery or energy harvesting technologies.
For example smart wearables, self sustaining sensors/actuators.
Often a single chip handles the whole application logic and data transmission.
For example a single Bluetooth chip is used in a smart watch to transmit data to a smart phone.
The chip also react and shows the user necessary data of the application use case.
As a result more energy saving wireless protocols are developed like ZigBee\footnote{\url{http://www.zigbee.org/}}, Bluetooth\footnote{\url{https://www.bluetooth.com/}} or LoRa\footnote{\url{https://www.lora-alliance.org/}}.

We assume in the future, IoT devices use a lot of different networking technologies which are more energy efficient than TCP/IP based networks.
But traditional IoT messaging standards like HTTP and MQTT\cite{eclipseiotdevelopersurveyresults} provides well understood paradigms.

To mix the best of two worlds the well understood publish-subscribe paradigm and energy efficient networking technologies should be used together.
We suggest MQTT-SN as a non connection oriented protocol which is easy to implement for most constrained devices.
To connect traditional IoT messaging standards and application-layer protocols we provide a prototypical MQTT-SN gateway implementation. During the implementation we want to learn and review which problems we needed and will need to solve for future developments and which problems we already solved.
 
This project report is structured as follows.
In \autoref{sec:relatedwork} we give a definition of different device types and provide a overview over the two most used messaging protocols HTTP, MQTT and MQTT-SN.
Then we justify why we think MQTT-SN is well suited for so called constrained devices.
Afterwards the MQTT-SN architecture is introduces.
In \autoref{sec:nonfunctionallimitationsandrequirements} we provide the target hardware and software environment for our project, as well as define non-functional requirements and limitations.
The MQTT-SN gateway enviroment independent Core Components are introduces in \autoref{sec:coregatewaycomponents} and implement on Linux in \autoref{sec:linuxgatewayimplementation}. Through a example MQTT-SN client life cycle MQTT-SN procedures we identified gateways procedures.
Then we provide a Linux gateway implementation for the Core Components in \autoref{sec:linuxgatewayimplementation}. This implementation can also used with minor changes on the target hardware and software environment.
Next in \autoref{sec:unitandregressiontesting} we implemented a automated unit and regression test environment.
Afterwards in \autoref{sec:BluetoothLowEnergySocket} we exchange the UDP socket implementation by a Bluetooth 4.0 Low Energy implementation.
Last we conclude our project and provide suggestions for future work.